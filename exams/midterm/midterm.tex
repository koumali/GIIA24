\documentclass[palatino,code]{ensaexam}

\usepackage{minted}
\usepackage{dirtree}
% \setlength{\DTbaselineskip}{20pt}
\DTsetlength{1em}{3em}{0.5em}{1pt}{3pt}
\begin{document}
\ModuleName{Introduction a Linux}
\ExamCode{GIIA24}
\ExamPeriod{Spring 2022}
\TimeAllowed{60}
\Logo{
\begin{center}
  \includegraphics[width=3cm, height=3cm]{ENSA-SAFI.png}
\end{center}
}
\Instructions{
  \begin{itemize}
    \item Vous avez {\bf\TheTimeAllowed\; minutes}. 
    \item Vérifier que vous disposez de toutes les pages. 
    \item L'échange d'outils est strictement \textbf{interdit}.
  \end{itemize}
  
}
\MakeHeading
\vspace*{1cm}

%% For questions use the command 
%% \begin{questions} 
%% \questoin[grade]
%% \end{questions}


\begin{questions}
% Questions de cours {{{ %
\titledquestion{Théorie}
\begin{parts}
  \part[1] Donner la  définition de \textbf{Linux BSD}. 
  \part[2] Donner les quatre principes du \textbf{logiciels libres}.
  \part[1] Donner deux fonctionnalités principales d'un \textbf{gestionnaire de
  session}.
  \part[1] Quel est le nom du service responsable d'affichage graphique dans
  Linux.
\end{parts}

% }}} Questions de cours %
\vspace*{1cm}
% Arborescence {{{ %
\titledquestion{Arborescence}
Dans cet exercice, on cherche a créer l'arborescence suivante:\\

  \dirtree{%
    .1 Disque.
    .2 Marie.
    .3 Octobre.
    .4 devoir.tex.
    .3 Septembre.
    .4 devoir2.tex.
    .3 Physique.
    .4 devoir3.tex.
    .2 Pierre.
    .3 headers.
    .4 ball.hpp.
    .4 player.hpp.
    .3 srcs.
    .4 ball.cpp.
    .4 player.cpp.
    .3 Makefile.
  }

\begin{parts}
  \part[2] Donner les commandes le contenu du dossier \textbf{Disque}. 
  \part[1] Quelle est la commandes pour afficher ce contenu sans les fichier
  fichiers d'extensions \textbf{.tex, .cpp et .hpp}.
  \part[1] Afficher le contenu de ce dossier par ordre \textbf{décroissant} de
la \textbf{date de modification}.
\part[1] Donner la commande pour le fichier \texttt{devoir.tex}  dans le dossier
\textbf{srcs}.
\end{parts}

\vspace*{1cm}

% }}} Arborescence %
% Commands I {{{ %
% test the following commands
%   find ,  locate , date, history
\titledquestion{Commandes I}

On suppose que vous avec un projet appelle \textbf{Gestion Classes} qui gère
les notes et les informations d'une classe. Ce dossier contient des dossiers de
structurations et des fichiers \texttt{java} et des fichiers de configurations.

\begin{parts}
  \part[1] Donner la commande qui vous donne le  \textbf{ nombre de dossiers}  de ce
  projet.
  \part[1] Vous savez que ce projet contient un fichier \textbf{java} qui commence avec
  \textbf{student}, donner la commande pour afficher le chemin de ce fichier.
  \part[2] Aussi vous informe qu'il y as un fichier qui s'appelle
  \texttt{studentsDates.txt} qui contient sur \textbf{chaque ligne} la date de
  naissance d'un étudiant.
  \begin{itemize}
    \item Donner la commande pour trouver ce fichier.
    \item Modifier votre ancienne commande pour donner le nombre des étudiants.
  \end{itemize}
  \part[2] Donner la commande pour afficher le nom des mois de naissance
  (\textbf{sans répétition}) de tous les étudiants.
  Par exemple si ce fichier contient les trois dates:
  \begin{verbatim}
   2019-06-12
   2020-05-05   
   2020-06-10
   2019-05-31
   2020-05-05   
  \end{verbatim}
  
  La commande affichera 
  \begin{verbatim}
   Mai 
   Juin
  \end{verbatim}
 Puisque ces les deux mois disponibles. 
\end{parts}
\newpage
% }}} Commands I %
% Commandes II {{{ %
\titledquestion{Commandes II}
Une compagnie de ventre d'articles sportifs enregistre ces achats dans un fichier
\textbf{csv}(comma separated values). Le fichier contient le nom des
articles vendus avec leur catégories.  Un exemple de fichier est:

\begin{verbatim}
 haltere,Musculation,34
 Skateboar,Pied,104
 Barre a disque,Musculation,80
 gants de box,protection,90
 .
 .
 .
\end{verbatim}


La premier colonne représente le nom du l'article par exemple \texttt{haltere},
la deuxième est la catégorie de l'article comme \texttt{Musculation} et
finalement la troisième est le nombre de vente \texttt{34}.

\begin{parts}
  \part[2] Donner la commande pour afficher les articles tries selon l'ordre
  décroissant de leurs nombre de vente.
  \part[2] Donner la nom des  \textbf{cinq} articles les plus vendues.
\end{parts}


% Uniq and sort 
% }}} Commandes II %
\end{questions}
 
 
\end{document}

